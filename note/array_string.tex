\chapter{数组与字符串}
\section{数组的定义及数组元素的引用}
    \paragraph{数组的定义}
        \begin{lstlisting}[language = {C}, gobble = 12]
            <element_type> <array_name>[<array_length>];
        \end{lstlisting}
    \textbf{Notation :}
    \begin{enumerate}
        \item 数组索引从 0 开始
        \item 通过 \Code{sizeof(<array\_name>)} 来获取数组的长度
    \end{enumerate}

    \subsection{二维和多维数组}
    \paragraph{数组存储规则}
        对于多维数组, 其存储规则可以概括为: 序号越靠右, 存储优先级越高. 此外, 若将数组的序号看成是一个多进制数, (e.g., \Code{[8][9][2]} 看成数字892), 并记该数为 $a$, 则存储顺序是按照 $a$ 增加的顺序来的.
